\documentclass[aspectratio=43]{beamer}%[handout]
\mode<handout>{
\usepackage{pgfpages}
\pgfpagesuselayout{4 on 1}[letterpaper,border shrink=5mm, landscape]
}
\usepackage[T1]{fontenc}
\usepackage[utf8]{inputenc}
\usepackage[spanish]{babel}
\usepackage{beamerthemeshadow,beamerthemesplit,cite,cancel,lmodern,eso-pic,fancyvrb,textcomp,lmodern,url,times,booktabs,amssymb,amsmath,ragged2e,float,subfig,xspace,epic,eepic,multicol,multirow,colortbl,color,graphicx,url}
\usepackage[normalem]{ulem} %tachar texto con \sout{}
\usepackage{listings} %CODIGOs
\usepackage[sfdefault]{roboto}
\setcounter{tocdepth}{3}
\usepackage[figurename=]{caption}
%Colores Ulagos
\definecolor{gray97}{gray}{.97}
\definecolor{gray75}{gray}{.75}
\definecolor{gray45}{gray}{.45}
\definecolor{listinggray}{gray}{0.9}
\definecolor{lbcolor}{rgb}{0.9,0.9,0.9}
%Colores Ulagos
\definecolor{amarillo}{RGB}{255,182,18}
\definecolor{verde}{RGB}{52,178,51}
\definecolor{rojo}{RGB}{237,41,57}
\definecolor{azulu}{RGB}{19,15,204}
\definecolor{azul}{RGB}{1,110,185}
\definecolor{negro}{RGB}{35,31,32}
\definecolor{naranjo}{RGB}{251,79,20}
\newcommand\rojo[1]{\textcolor[RGB]{237,41,57}{#1}}
\newcommand\gris[1]{\textcolor[gray]{.65}{#1}}
\newcommand\azul[1]{\textcolor[RGB]{19,15,204}{#1}}
\newcommand\verde[1]{\textcolor[RGB]{5,101,99}{#1}}
\newcommand\naranjo[1]{\textcolor[RGB]{251,79,20}{#1}}

%%%%%%%%%%%%%%%%%%%%%%%%%
%% Tikz
%%%%%%%%%%%%%%%%%%%%%%%%%
\usepackage{tikz}
\usetikzlibrary{trees}
\usetikzlibrary{snakes}
\usetikzlibrary{arrows}
\usetikzlibrary{shapes}
\usetikzlibrary{backgrounds}
\usetikzlibrary{patterns}
\usetikzlibrary{fit}
\usetikzlibrary{positioning} % LATEX and plain TEX
%%%%%%%%%%%%%%%%%%%%%%%%%%

%%%%%%%%%%%%%%%%%%%%%TEMA
\mode<presentation>
\usetheme{CambridgeUS}
\usecolortheme[named=azul]{structure}
\useinnertheme{rectangles}    
\useoutertheme{infolines} 
               
\setbeamercovered{transparent}

\setbeamerfont{block title}{size={}}
\setbeamerfont{title}{shape=\scshape}
\setbeamerfont{frametitle}{shape=\scshape}
\setbeamerfont{author}{shape=\scshape}
\setbeamerfont{institute}{shape=\scshape}
\setbeamerfont{block title}{shape=\scshape}

\setbeamertemplate{blocks}[rounded][shadow=true]
\setbeamertemplate{itemize items}[default]
\setbeamertemplate{enumerate items}[circle]
\setbeamertemplate{description item}[align left]
\setbeamertemplate{blocks}[rounded][shadow=true]
\setbeamertemplate{title page}[default][colsep=-4bp,rounded=false,shadow=false]
%\setbeamertemplate{frametitle continuation}

\setbeamercolor*{palette primary}{use=structure,fg=azul,bg=listinggray}
\setbeamercolor*{palette secondary}{use=structure,fg=azul,bg=gray97}
\setbeamercolor*{palette tertiary}{use=structure,fg=gray97,bg=azul}
\setbeamercolor*{palette sidebar primary}{use=structure,fg=azul}
\setbeamercolor*{palette sidebar tertiary}{use=structure,fg=azul}
\setbeamercolor*{title}{use=structure,fg=white}
\setbeamercolor*{author}{use=structure,fg=white}
\setbeamercolor*{institute}{use=structure,fg=white}
\setbeamercolor{frametitle right}{bg=azul!50!white}
\setbeamercolor{structure}{fg=azul}
\setbeamercolor{block title}{use=structure,fg=white,bg=amarillo}
\setbeamercolor{block body}{use=structure,fg=negro,bg=amarillo!20!white}
\setbeamercolor{block title example}{use=structure,fg=white,bg=verde!80!black}
\setbeamercolor{block body example}{use=structure,fg=negro,bg=verde!20!white}
\setbeamercolor{block title alerted}{use=structure,fg=white,bg=rojo!90!negro}
\setbeamercolor{block body alerted}{use=structure,fg=negro,bg=rojo!20!white}
%%%%%%%%%%%%%%%%%%%%%FIN TEMA

%lstlisting %%%%%CODIGOS
\lstset{tabsize=4,language=C,basicstyle=\small,upquote=true,aboveskip={1.5\baselineskip},columns=fixed,showstringspaces=false, extendedchars=true,breaklines=true, prebreak = \raisebox{0ex}[0ex][0ex]{\color{gray75}{\ensuremath{\hookleftarrow}}},showtabs=false,showspaces=false,showstringspaces=false,identifierstyle=\ttfamily,keywordstyle=\color[rgb]{0,0,1}, commentstyle=\color[rgb]{0.133,0.545,0.133}, stringstyle=\color[rgb]{0.627,0.126,0.941}} 
  
%%%%%%%%%%%%%%PORTADA
\title[Primer documento en \LaTeX{}]{\textbf{\huge{Primer documento en \LaTeX{}}}\vspace{-0.3cm}}
\subtitle{Formación General\vspace{-0.7cm}}
\author[Juan José Ramírez Lama]{\small{\textbf{Juan José Ramírez Lama} \\ \texttt{juan.ramirez@ulagos.cl} }\vspace{-0.2cm}}
\institute[ULA]{\small{\textbf{Departamento de Ciencias Exactas}\\Ingeniería Civil en Informática}\vspace{-0.5cm}}
\date[\today]{}
%%%%%%%%%%%%%FIN PORTADA

\begin{document}
\setbeamertemplate{background}{\includegraphics[height=9.2cm,width=12.8cm]{fondoula43}}%4:3
%\setbeamertemplate{background}{\centering\includegraphics[height=8.6cm,width=16.1cm]{fondoula169}}%16:9


%Pagina de Portada
\begin {frame} [plain]
\vspace{5.25cm}
\titlepage
\end {frame}
\setbeamertemplate{background}{}
%%%%%%%%%%%%%%%%ÍNDICES
\section[Contenido]{}
\frame{
  \frametitle{\textbf{Contenido}}
\setcounter{tocdepth}{2}%1: solo titulo principal, 2: titulo y subtitulo, 3....
\scriptsize
\tableofcontents[]
}%Generacion de Indice por capitulo
\AtBeginSection[]{
\begin{frame}
\frametitle{\textbf{Contenido}}
\scriptsize
\tableofcontents[currentsection]
\end{frame}}

\AtBeginSubsection[]{
\begin{frame}
\frametitle{\textbf{Contenido}}  
\scriptsize
\tableofcontents[currentsection,currentsubsection]
\end{frame}
}
%%%%%%%%%%%%%%%%FIN ÍNDICES







%%%%%%%%%%%%%%%%%%%%%%%%%%%%%%%%%%%%%
%%%%%%%%%%%%%%%%%INICIO PRESENTACION
\section{Bases del Lenguaje \LaTeX{}}
\subsection{Formato de Archivos}

\begin{frame}[fragile]
\frametitle{\textbf{Formato de Archivos}}
\justifying
 \begin{exampleblock}{Archivos de Texto Plano}
\begin{itemize}\justifying
  \item Se escribe un archivo de texto plano que contiene el contenido del documento y etiquetas que lo estructuran.
  \item Separación de formato y contenido.
\end{itemize}

\end{exampleblock}

\end{frame}
\subsection{Comandos}
\begin{frame}[fragile]
\frametitle{\textbf{Comandos}}
\justifying
 \begin{exampleblock}{Formato de los comandos}
\begin{itemize}\justifying
  \item \verb+\comando[opciones{parametros}]+
  \item [Ej.:] \verb+\includegraphics[width=6cm]{logoulagos.png}+
\end{itemize}

\end{exampleblock}

\begin{exampleblock}{Entornos}
\begin{itemize}\justifying
  \item Afectan a una sección de código.
  \item \verb+\begin{entorno}...\end{entorno}+
  \item [Ej.:] \verb+\begin{center} Texto Centrado \end{center}+
\end{itemize}

\end{exampleblock}


\end{frame}
\subsection{Caracteres y Espaciado}
\begin{frame}[fragile]
\frametitle{\textbf{Gestión del Espaciado}}
\justifying

\begin{itemize}\justifying
  \item \LaTeX{} ignora varios espacios en blanco seguidos (tabuladores, espacios, líneas en blanco, etc).
  \item Los párrafos se separan con una linea en blanco.
\end{itemize}
 \begin{exampleblock}{Ejemplo}
   \lstset{language=}%SQL,basicstyle=\small}
   \vspace{-0.7cm}
\begin{lstlisting}
Prueba            de       espaciado y contracci\'on        de     espacios.

Nuevo p\'arrafo tras linea en blanco.
\end{lstlisting}\vspace{-0.3cm}
\end{exampleblock}

\begin{exampleblock}{Resultado}
Prueba            de       espaciado y contracci\'on        de     espacios.

Nuevo p\'arrafo tras linea en blanco.
\end{exampleblock}

\end{frame}

\begin{frame}[fragile]
\frametitle{\textbf{Caracteres Reservados}}
\justifying
 \begin{itemize}\justifying
  \item Hay algunos caracteres especiales que están reservados para \LaTeX{}.
  \item \verb+ # $ % ^ & _ { } ~ \+
  \item No se pueden introducir directamente como carácter, hay que introducir su código.
\end{itemize}

\begin{exampleblock}{Ejemplo:}
   \vspace{-0.7cm}
\begin{lstlisting}
\# \$ \% \^{} \& \_ \{ \} \~{} \textbackslash
\end{lstlisting}\vspace{-0.3cm}

\end{exampleblock}

\begin{exampleblock}{Resultado:}
\# \$ \% \^{} \& \_ \{ \} \~{} \textbackslash
\end{exampleblock}



\end{frame}

\begin{frame}[fragile]
\frametitle{\textbf{Comantarios}}
\justifying
 \begin{itemize}\justifying
  \item Es posible introducir comentarios en el texto que no se visualizarán.
  \item Útil para dejarse notas, separar partes del documento, detectar errores...
  \item Carácter \%.
\end{itemize}

\begin{exampleblock}{Ejemplo}
   \vspace{-0.7cm}
\begin{lstlisting}
Esto es un %muy breve
texto para demostrar c\'omo insertar comentarios.
\end{lstlisting}\vspace{-0.3cm}

\end{exampleblock}

\begin{exampleblock}{Resultado}
Esto es un % muy breve
texto para demostrar c\'omo insertar comentarios.
\end{exampleblock}


\end{frame}

\begin{frame}[fragile]
\frametitle{\textbf{Acentos y Caracteres No Estándar}}
\justifying
 \begin{itemize}\justifying
  \item Se usan caracteres de 7 bit (estándar).
  \item Caracteres de 8 bit (acentos eñes, etc): mediante comandos especiales. Se guardan como carácteres de 7 bit y el compilador los interpreta.
  \item Ejemplo: \verb+Ram\'irez+
  \item ?`Por qué?, para mantener la portabilidad. Todos los sistemas interpretan estos caracteres de la misma forma.
  \item Aunque no es obligación usarlos en todos los editores.
\end{itemize}

\end{frame}

\begin{frame}[fragile]
\frametitle{\textbf{Acentos y Caracteres No Estándar}}
\justifying
 \begin{minipage}[l]{0.48\linewidth}
\begin{center}
Acentos Comunes

\begin{tabular}{|c|c|}\hline
  \verb+\'a+&\'a \\
  \verb+\'e+&\'e \\
  \verb+\'i+&\'i \\
  \verb+\'o+&\'o \\
  \verb+\'u+&\'u \\
  \verb+\~n+&\~n \\
  \verb+?`+&?` \\
  \verb+?+&? \\
  \verb+!` +&!`\\
  \verb+!+&! \\\hline
\end{tabular}
\end{center}


\end{minipage}\hfill
\begin{minipage}[r]{0.48\linewidth}
\begin{center}
Acentos en modo texto

\begin{tabular}{|c|c|}\hline
 \verb+\'{o}+ & \'{o}\\
 \verb+\`{o}+ & \`{o}\\
 \verb+\^{o}+ & \^{o}\\
 \verb+\"{o}+ & \"{o}\\
 \verb+\~{o}+ & \~{o}\\
 \verb+\~{n}+ & \~{n}\\
 \verb+\c{c}+ & \c{c}\\
 \verb+\={o}+ & \={o}\\
 \verb+\b{o}+ & \b{o}\\
 \verb+\.{o}+ & \.{o}\\
 \verb+\d{o}+ & \d{o}\\
 \verb+\r{o}+ & \r{o}\\
 \verb+\v{o}+ & \v{o}\\
 \verb+\t{oo}+ & \t{oo}\\
	\hline
\end{tabular}

\end{center}

\end{minipage}

\end{frame}

\begin{frame}[fragile]
\frametitle{\textbf{Acentos y Caracteres No Estántar}}
\justifying
 \begin{itemize}\justifying
  \item El paquete \textbf{inputenc} permite escribir esos caracteres directamente (es decir, á en lugar de \verb+\'a+).
  \item Se pierde cierta portabilidad.
  \item En realidad casi todos los editores (como \TeX{}Maker) permiten seleccionar la codificación.
  \item Posible solucion: Buscar y reemplazar.
  \item Hay editores (como Kile) que según escribes convierten los caracteres a la nomenclatura estándar.
\end{itemize}

\end{frame}

\begin{frame}[fragile]
\frametitle{\textbf{Estructura Básica de un Archivo}}
\justifying
 \begin{itemize}\justifying
  \item Se define el tipo de documento.
  \item Preámbulo, donde se cargan paquetes, se modifican comandos, etc.
  \item Cuerpo del documento.
\end{itemize}



\begin{exampleblock}{}
\begin{verbatim}
    \documentclass[letter,10pt]{article}
    \usepackage[utf8]{inputenc}
    \author{Juan Jos\'e Ram\'irez Lama}
    % Comienzo del documento
    \begin{document}
    % contenido del documento
    Texto del documento
    \end{document}
\end{verbatim}
\end{exampleblock}
\end{frame}

\begin{frame}[fragile]
\frametitle{\textbf{Incluir Archivos}}
\justifying
 \begin{itemize}\justifying
  \item \LaTeX{} permite que el documento se divida en varios archivos.
\begin{itemize}\justifying
  \item Útil para modularizar documentos grandes.
  \item Facilita la redacción colaborativa de documentos.
  \item Permite separar preámbulos y contenido.
\end{itemize}
  \item Comando \textbf{input}
  \item \verb+\input{documento}+
\end{itemize}
\begin{exampleblock}{}
{\scriptsize
\begin{verbatim}
    \documentclass[letter,10pt]{article}
    \usepackage[utf8]{inputenc}
    \author{Juan Jos\'e Ram\'irez Lama}
    % Comienzo del documento
    \begin{document}
    % contenido del documento
    \input{introduccion.tex}
    \input{formulas.tex}
    \input{bibliografia.tex}
    Texto del documento
    \end{document}
\end{verbatim}
}
\end{exampleblock}

\end{frame}

\begin{frame}[fragile]
\frametitle{\textbf{Clases de Documentos}}
\justifying
 \begin{description}\justifying
  \item[article] Para artículos de revistas científicas, congresos, informes breves, documentación de programas, invitaciones, etc.
  \item[proc] Para Proceedings.
  \item[report] para informes más largos que contengan varios capítulos como pequeños libros, tesis doctorales, proyectos de titulo, etc.
  \item[boot] para libros o incluso tesis más extensas.
  \item[beamer] para presentaciones.
\end{description}

\end{frame}

\section{Creando un Documento}
\subsection{Formato Básico}
\begin{frame}[fragile]
\frametitle{\textbf{Preparando el directorio en \LaTeX{}}}
\justifying
 \begin{itemize}\justifying
  \item En primer lugar, crea una carpeta para meter dentro el documento.
  \item Dentro irá el archivo principal .tex, las imágenes y cualquier otro archivo que desees incluir.
  \item y también los archivos que se generan al compilar.
\end{itemize}

\end{frame}

\begin{frame}[fragile]
\frametitle{\textbf{Empezando a escribir en \LaTeX{}}}
\justifying
 \begin{itemize}\justifying
  \item Escribimos un documento con el texto indicado abajo.
  \item Lo guardamos en un directorio conocido con extensión .tex.
  \item Compilamos y vemos el resultado.
%  \item Si quieres, puedes probar a compilar con \LaTeX{}, visualizar el .dvi y despues convertir a otros formatos
\end{itemize}

\begin{exampleblock}{}
\begin{verbatim}
    \documentclass[letter,10pt]{article}
    % Comienzo del documento
    \begin{document}
    % contenido del documento
    HOLA MUNDO
    \end{document}
\end{verbatim}
\end{exampleblock}
\end{frame}

\begin{frame}[fragile]
\frametitle{\textbf{Datos del Documento}}
\justifying
 \begin{itemize}\justifying
  \item Añadimos algunos datos al documento: título, autor y fecha.
  \item Lo escribimos en el preámbulo (antes del \verb+\begin{document}+)
  \item Compilamos de nuevo para ver el resultado.
\end{itemize}

\begin{exampleblock}{}
\begin{verbatim}
    \documentclass[letter,10pt]{article}
    \title{Apuntes curso \LaTeX{}}
    \author{Juan Jos\'e Ram\'irez Lama}
    \date{Octubre 2015}
    % Comienzo del documento
    \begin{document}
    % contenido del documento
    Hola Mundo
    \end{document}
\end{verbatim}

\end{exampleblock}
\end{frame}

\begin{frame}[fragile]
\frametitle{\textbf{Datos del Documento}}
\justifying
 \begin{itemize}\justifying
  \item ?`Todo igual?, Hay que decirle a \LaTeX{} que genere un título: \verb+\maketitle+.
  \item Esto va dentro del documento (es título es parte del contenido).
\end{itemize}
\begin{exampleblock}{}
\begin{verbatim}
    \documentclass[letter,10pt]{article}
    \title{Apuntes curso \LaTeX{}}
    \author{Juan Jos\'e Ram\'irez Lama}
    \date{Octubre 2015}
    % Comienzo del documento
    \begin{document}
    \maketitle
    % contenido del documento
    Hola Mundo
    \end{document}
\end{verbatim}
\end{exampleblock}

 \begin{block}{}
Nada de lo que va antes de \verb+\begin{document}+ se muestra, son solo sentencias de control.
\end{block}
\end{frame}


\begin{frame}[fragile]
\frametitle{\textbf{Soporte para acentos}}
\justifying
 \begin{itemize}\justifying
  \item ?`Faltan los tildes?
  \item Incluir el paquete \textbf{inputenc} para soporte de tildes.
  \item No es parte del contenido: en el preámbulo.
\end{itemize}
\begin{exampleblock}{}
\begin{verbatim}
    \documentclass[letter,10pt]{article}
    \usepackage[utf8]{inputenc}
    \title{Apuntes curso \LaTeX{}}
    \author{Juan Jos\'e Ram\'irez Lama}
    \date{Octubre 2015}
    % Comienzo del documento
    \begin{document}
    \maketitle
    % contenido del documento
    Hola Mundo
    \end{document}
\end{verbatim}
\end{exampleblock}
\end{frame}

\begin{frame}[fragile]
\frametitle{\textbf{Insertando un abstract (resumen)}}
\justifying
 \begin{itemize}\justifying
  \item Entorno abstract:
  \item[] \verb+\begin{abstract}...\end{abstract}+
\end{itemize}\begin{exampleblock}{}
\begin{verbatim}
    \begin{document}
    \maketitle
    % contenido del documento
    
    \begin{abstract}
        Texto del abstract
    \end{abstract}
    
    Hola Mundo
    \end{document}
\end{verbatim}
\end{exampleblock}

\end{frame}
\subsubsection{División del documento}
\begin{frame}[fragile]
\frametitle{\textbf{División del documento: Capítulos, secciones, ...}}
\justifying
 \begin{itemize}\justifying
  \item Para estructurar el documento, hay que dividirlo en secciones y subsecciones.
  \item Se usan órdenes de \LaTeX{} que toman el título de la sección como argumento.
\end{itemize}
Las divisiones disponibles dependen del tipo de documento.
\begin{itemize}\justifying
  \item Clase \rojo{article}:
  \begin{itemize}\justifying
  \item \verb+\section{}+, \verb+\subsection{}+, \verb+\subsubsection{}+
  \item \verb+\paragraph{}+, \verb+\subparagraph{}+
  \item División que no afecta a la numeración de secciones: \verb+\part{}+.
\end{itemize}

  \item Clase \rojo{report} y \rojo{book}:
  \item \begin{itemize}\justifying
  \item Otra división adicional: \verb+\chapter{}+.
\end{itemize}

\end{itemize}

\end{frame}

\begin{frame}[fragile]
\frametitle{\textbf{División del documento: Capítulos, secciones, ...}}
\justifying
 Comandos \verb+\chapter{}+, \verb+\section{}+ y \verb+\subsection{}+.
 \begin{exampleblock}{Para Practicar}
\begin{itemize}\justifying
  \item Inserta el capítulo: Creación de documentos en \LaTeX{}.
  \item Inserta las siguientes secciones: Introducción y Principios Básicos.
  \item Inserta dentro de la ultima sección la subsección: Comandos y Entornos \LaTeX{}.
\end{itemize}

\end{exampleblock}

\end{frame}

\begin{frame}[fragile]
\frametitle{\textbf{División del documento: Capítulos, secciones, ...}}
\justifying
 Comandos \verb+\chapter{}+, \verb+\section{}+ y \verb+\subsection{}+.
 \begin{exampleblock}{}

\begin{verbatim}
\begin{document}
\maketitle
% Contenido
\chapter{Creación de documentos en \LaTeX{}}
\section{Introducción}
\section{Principios Básicos}
\subsection{Comandos y Entornos \LaTeX{}}

Hola Mundo
\end{document}
\end{verbatim}

\end{exampleblock}

\end{frame}

\begin{frame}[fragile]
\frametitle{\textbf{División del documento: Capítulos, secciones, ...}}
\justifying
 \begin{block}{Capítulos tipo Apéndice}
\begin{itemize}\justifying
  \item El comando \verb+\appendix+ hace que los siguientes capítulos se numeren como apéndices (con letras).
  \item Si estamos usando la clase article, cambia la numeración de secciones.
\end{itemize}

\end{block}

\end{frame}

\begin{frame}[fragile]
\frametitle{\textbf{División del documento en varios archivos}}
\justifying
 
 \begin{itemize}\justifying
  \item El comando \verb+input+ permite introducir en el documento texto escrito en otro archivo .tex
  \item Se puede usar para separa diferentes partes del contenido (como capítulos de una tesis), o para separar preámbulo y contenido (tener un archivo a parte para configuración e importarlo).
\end{itemize}

 
\end{frame}

\begin{frame}[fragile]
\frametitle{\textbf{División del documento en varios archivos}}
\justifying
 \begin{alertblock}{Para Practicar}
\begin{itemize}
  \item Crea en el mismo directorio otro archivo .tex con contenido (no se debe escribir de nuevo \verb+\begin{document}+), solo contenido.
  \item Importalo desde el documento principal usando \verb+\input{nombreArchivo}+.
  \item Recuerda no usar tildes ni espacios en el nombre del archivo.
\end{itemize}

\end{alertblock}

\end{frame}

\begin{frame}[fragile]
\frametitle{\textbf{Generando un Índice}}
\justifying
 \begin{itemize}\justifying
  \item Comando \verb+\tableofcontents+
  \item Necesario compilar varias veces para que asigne la numeración correctamente.
\end{itemize}

\begin{exampleblock}{}
\begin{verbatim}
\begin{document}
\maketitle
\tableofcontents

% Contenido
\chapter{Creación de documentos en \LaTeX{}}
\section{Introducción}
\section{Principios Básicos}
\subsection{Comandos y Entornos \LaTeX{}}

Hola Mundo
\end{document}
\end{verbatim}
\end{exampleblock}


\end{frame}
\begin{frame}[fragile]
\frametitle{\textbf{Selección de Idioma}}
\justifying
 \begin{itemize}\justifying
  \item Tenemos palabras en inglés: Chapter, Table of Contents, etc.
  \item Cambiar el idioma con el paquete babel
  \item \verb+\usepackage[spanish]{babel}+
\end{itemize}

\begin{exampleblock}{}\scriptsize
\begin{verbatim}
    \documentclass[letter,10pt]{article}
    \usepackage[spanish]{babel}
    
    \usepackage[utf8]{inputenc}
    \title{Apuntes curso \LaTeX{}}
    \author{Juan Jos\'e Ram\'irez Lama}
    \date{Octubre 2015}
    % Comienzo del documento
    \begin{document}
    \maketitle
    \listofcontents
    % contenido del documento
    \chapter{Creación de documentos en \LaTeX{}}
    \section{Introducción}
    \section{Principios Básicos}
    \subsection{Comandos y Entornos \LaTeX{}}
    Hola Mundo
    \end{document}
\end{verbatim}
\end{exampleblock}


\end{frame}

\begin{frame}[fragile]
\frametitle{\textbf{Notas al Pie}}
\justifying
 Comando \verb+\footnote{}+
 
  \begin{exampleblock}{Ejemplo}
\verb+Hola Mundo\footnote{Palabra usada comúnmente al+
\verb+ crear un primer programa o documento}+
\end{exampleblock}
 
 \begin{exampleblock}{Resultado}
Hola Mundo\footnote{Palabra usada comúnmente al crear un primer programa o documento}
\end{exampleblock}
\end{frame}

\begin{frame}[fragile]
\frametitle{\textbf{Creación de Enlaces}}
\justifying
 \begin{itemize}\justifying
  \item Usando el comando \verb+\url{}+ podemos crear enlaces.
  \item Incluir en el preámbulos los paquetes \verb+url+ y \verb+hyperref+
  \item \verb+hyperref+ crea enlaces internos, por ejemplo, en el índice y los pies de página.
  \item Los enlaces se muestran recuadrados en el pdf, pero esos recuadros no se verán al imprimir el documento.
\end{itemize}
\begin{block}{}
   \lstset{language=}%SQL,basicstyle=\small}
   \vspace{-0.7cm}
\begin{lstlisting}
\usepackage{url}
\usepackage[breaklinks=true]{hyperref}
\section{Introducción}
Las presentaciones están en PLATEA\footnote{\url{http://platea-osorno.ulagos.cl}}.
\end{document}
\end{lstlisting}\vspace{-0.3cm}




\end{block}

\end{frame}

\begin{frame}[fragile]
\frametitle{\textbf{Creación de Enlaces}}
\justifying
 \begin{itemize}\justifying
  \item También podemos crear enlaces con un texto diferente a la dirección del enlace.
  \item Para ello usaremos el comando \verb+\href+, de la siguiente forma:
  \item []\verb+\href{pagina a enlazar}{texto enlace}+
\end{itemize}

\begin{exampleblock}{ejemplo}
\verb+\href{http://www.juaramir.com}{Mi Página Personal}+
\end{exampleblock}
\begin{block}{}
\href{http://www.juaramir.com}{Mi Página Personal}
\end{block}




\end{frame}

\subsection{Formato de palabras y párrafos}

\begin{frame}[fragile]
\frametitle{\textbf{Enfatizar Palabras}}
\justifying
 \begin{description}\justifying
  \item[emph] Para enfatizar palabras, de acuerdo al texto. \emph{Recomendado}.
  \item [textbf] Para texto en \textbf{Negrita}
  \item [textit] Para tecto en \textit{Cursiva}
  \item [underline] Para texto \underline{subrayado}
  \item [texttt] Para texto estilo \texttt{máquina de escribir}
  \item [textsf] Para texto \textsf{Sans-Serif}
\end{description}

\end{frame}

\begin{frame}[fragile]
\frametitle{\textbf{Alineación de Párradfo}}
\justifying
 \begin{description}\justifying
  \item [flushleft] Entorno para alinear el texto a la izquierda.
  \item [flushright] Entorno para alinear el texto a la derecha.
  \item [center] Entorno para centrar un párrafo, tabla, una figura, etc.
\end{description}

\begin{exampleblock}{Ejemplos}
\begin{verbatim}
\begin{flushleft}Texto Izquierda\end{flushleft}
\begin{flushright}Texto Derecha\end{flushright}
\begin{center}Texto Centrado\end{center}
\end{verbatim}
\end{exampleblock}

\begin{block}{}
\begin{flushleft}Texto Izquierda\end{flushleft}
\begin{flushright}Texto Derecha\end{flushright}
\begin{center}Texto Centrado\end{center}
\end{block}


\end{frame}

\subsection{Otros entornos útiles}
\begin{frame}[fragile]
\frametitle{\textbf{Algunos entornos útiles: verbatim}}
\justifying
 \begin{itemize}\justifying
  \item El entorno verbatim permite introducir código que no sea interpretado. 
  \item Para usarlo dentro de un párrafo: \verb+\verb+++
\end{itemize}
\begin{exampleblock}{}
\begin{verbatim}
Para enfatizar texto se usa el comando:
 \verb+\emph{}+.
Por ejemplo:
\verb+\emph{esto va enfatizado}+
\end{verbatim}
\end{exampleblock}

\begin{block}{}
Para enfatizar texto se usa el comando \verb+\emph{}+.
Por ejemplo:
\verb+\emph{esto va enfatizado}+
\end{block}



\end{frame}
\begin{frame}[fragile]
\frametitle{\textbf{Entorno verbatim}}
\justifying
 \begin{itemize}\justifying
  \item En párrafo separado:
  \item []\verb+\begin{verbatim}...\end{verbatim}+

\end{itemize}
\begin{exampleblock}{}
   \lstset{language=}%SQL,basicstyle=\small}
   \vspace{-0.7cm}
\begin{lstlisting}
\begin{verbatim}
Se pueden crear listas como sigue:
\begin{itemize}
  \item item 1
  \item item 2
\end{itemize}
\end{verbatim}
\end{lstlisting}\vspace{-0.3cm}
\end{exampleblock}

\begin{block}{}
\begin{verbatim}
Se pueden crear listas como sigue:
\begin{itemize}
  \item item 1
  \item item 2
\end{itemize}
\end{verbatim}
\end{block}


\end{frame}

\begin{frame}[fragile]
\frametitle{\textbf{Enumeración, Listas, y Descripción}}
\justifying
\begin{description}\justifying
  \item [enumerate] Permite crear listas numeradas
  \item [itemize] Permite crear listas no numeradas
  \item [description] Permite crear listas de definiciones
\end{description}
\end{frame}

\begin{frame}[fragile]
\frametitle{\textbf{Enumeración}}
\justifying
 
 \begin{block}{Ejemplo}
\begin{verbatim}
\begin{enumerate}
\item Naranjas
\item Manzanas
\item Bananas
\end{enumerate}
\end{verbatim}

\end{block}

\begin{block}{Resultado}
\begin{enumerate}
\item Naranjas
\item Manzanas
\item Bananas
\end{enumerate}
\end{block} 
\end{frame}

\begin{frame}[fragile]
\frametitle{\textbf{Listas}}
\justifying
 \begin{exampleblock}{Ejemplo}
\begin{verbatim}
\begin{itemize}
\item Tomates
\item Pepinos
\item Cebollas
\end{itemize}
\end{verbatim}

\end{exampleblock}

\begin{block}{Resultado}
\begin{itemize}
\item Tomates
\item Pepinos
\item Cebollas
\end{itemize}
\end{block}

\end{frame}

\begin{frame}[fragile]
\frametitle{\textbf{Descripción}}
\justifying
 \begin{exampleblock}{Ejemplo}
\begin{verbatim}
\begin{description}
  \item [Chocolate] Elaborado a base de cacao
  \item [Caramelo] Elaborado a base de az\'ucar
\end{description}
\end{verbatim}

\end{exampleblock}

\begin{block}{Resultado}
\begin{description}
  \item [Chocolate] Elaborado a base de cacao
  \item [Caramelo] Elaborado a base de az\'ucar
\end{description}

\end{block}

\end{frame}

\begin{frame}[fragile]
\frametitle{\textbf{Cambiar el formato de la Enumeración}}
\justifying
 Por ejemplo, para que la numeración de segundo nivel sea de la forma 1.1.
 
 \begin{block}{}
   \lstset{language=}%SQL,basicstyle=\small}
   \vspace{-0.7cm}
\begin{lstlisting}
\renewcommand{\theenumii}{\arabic{enumii}}
\renewcommand{\labelenumii}{\theenumi .\theenumii .}
\end{lstlisting}\vspace{-0.3cm}

\end{block}

\end{frame}

\begin{frame}[fragile]
\frametitle{\textbf{Citas y Poemas}}
\justifying
 \begin{description}\justifying
  \item [quote] permite escribir citas, frases importantes o ejemplos.
  \item [quotation] es útil para citas largas que ocupen varios párrafos.
  \item [verse] para escribir versos.
\end{description}

\end{frame}

\begin{frame}[fragile]
\frametitle{\textbf{quote}}
\justifying
 \begin{exampleblock}{ejemplo}
   \lstset{language=}%SQL,basicstyle=\small}
   \vspace{-0.7cm}
\begin{lstlisting}
Una regla tipogr\'afica para la longitud de la l\'inea es la siguiente:
\begin{quote}
    De media, ninguna l\'inea deber\'ia superar los 66 caracteres. 
\end{quote}
Por eso las p\'aginas \LaTeX{} tienen por defecto m\'argenes tan amplios.
\end{lstlisting}\vspace{-0.3cm}

\end{exampleblock}

\begin{block}{Resultado}
Una regla tipogr\'afica para la longitud de la l\'inea es la siguiente:
\begin{quote}
    De media, ninguna l\'inea deber\'ia superar los 66 caracteres. 
\end{quote}
Por eso las p\'aginas \LaTeX{} tienen por defecto m\'argenes tan amplios.
\end{block}


\end{frame}


\begin{frame}[fragile]
\frametitle{\textbf{verse}}
\justifying
 \begin{exampleblock}{ejemplo}
   \lstset{language=,basicstyle=\scriptsize}
   \vspace{-0.7cm}
\begin{lstlisting}
\begin{verse}
    \'Erase de un marinero\\
    que hizo un jard\'in junto al mar,\\
    y se met\'io a jardinero.\\
    Estaba en el jard\'in en flor,\\
    y el jardinero se fue\\
    por esos mares de Dios.
\end{verse}
\end{lstlisting}\vspace{-0.3cm}

\end{exampleblock}

\begin{block}{Resultado}
\begin{verse}
    \'Erase de un marinero\\
    que hizo un jard\'in junto al mar,\\
    y se met\'io a jardinero.\\
    Estaba en el jard\'in en flor,\\
    y el jardinero se fue\\
    por esos mares de Dios.
\end{verse}
\end{block}


\end{frame}

\begin{frame}[fragile]
\frametitle{\textbf{Theorem}}
\justifying
 \begin{itemize}\justifying
  \item Permite insertar sentencias separadas del texto y con números identificadores
  \item Se pueden usar para cualquier cosa que nos interese.
  \item Requiere el paquete \texttt{amsthm}.
  \item Se visualiza diferente en un documento y en una presentación.
\end{itemize}

\begin{exampleblock}{Ejemplo}
\begin{verbatim}
  \newtheorem{midef}{Definici\'on}
\begin{midef}
    Esto es una definici\'on.
\end{midef}
\end{verbatim}

\end{exampleblock}


\begin{block}{}
\newtheorem{midef}{Definici\'on}
\begin{midef}
    Esto es una definici\'on.
\end{midef}
\end{block}

\end{frame}

\begin{frame}[fragile]
\frametitle{\textbf{Theorem}}
\justifying
 \begin{itemize}\justifying
  \item El comando \verb+\newtheorem+ solo se pone una vez en todo el documento, para crear un nuevo tipo de \texttt{theorem} y asignarle un nombre.
  \item Cada vez que se quiera usar, se pone:
  \item [] \verb+\begin{nombre}...\end{nombre}+
  \item siendo \texttt{nombre} el nombre que le hemos asignado.
\end{itemize}
\end{frame}

\subsection{Referencias Cruzadas}\label{subsec:refcruzada}
\begin{frame}[fragile]
\frametitle{\textbf{Referencias Cruzadas}}
\justifying
 \begin{itemize}\justifying
  \item Permiten referenciar secciones, figuras, páginas, etc.
  \item \LaTeX{} maneja las referencias de forma muy efectiva.
  \item [] \verb+\label{etiqueta}+, \verb+\ref{etiqueta}+, \verb+pageref{etiqueta}+
\end{itemize}

\begin{block}{Ejemplo}
   \lstset{language=}%SQL,basicstyle=\small}
   \vspace{-0.7cm}
\begin{lstlisting}
\subsection{Referencias cruzadas}\label{subsec:refcruzada}

Como se ha descrito en la Subsecci\'on \ref{subsec:refcruzada}.
\end{lstlisting}\vspace{-0.3cm}

\end{block}


\begin{block}{}
Como se ha descrito en la Subsecci\'on \ref{subsec:refcruzada}.
\end{block}

\end{frame}

\begin{frame}[fragile]
\frametitle{\textbf{Referencias Cruzadas}}
\justifying
 \begin{description}\justifying
  \item [$\textbackslash$label\{etiqueta\}] Asigna una etiqueta a una sección, subsección, figura, tabla, fórmula, etc. Se suele indicar el tipo de objeto al que se referencia (sec, fig, tab).
  \item [$\textbackslash$ref\{etiqueta\}] Inserta una referencia al objeto con la etiqueta indicada. La referencia consiste en el número de sección, figura, etc.
  \item [$\textbackslash$pageref\{etiqueta\}] Inserta la página en que se encuentra la referencia indicada.
  \item [$\textbackslash$nameref\{etiqueta\}] Introduce el nombre del objeto que se quiere referenciar. Requiere la inclusión de paquete \texttt{nameref}.
\end{description}

\end{frame}

\subsection{Figuras y Tablas}
\begin{frame}[fragile]
\frametitle{\textbf{Tablas}}
\justifying
 \begin{itemize}\justifying
  \item Entorno tabular
  \item [] \verb+\begin{tabular}[pos]{formato}+
\item[] \verb+Tabla \end{tabular}+    
\end{itemize}
\begin{exampleblock}{Ejemplo}
\begin{minipage}[l]{0.48\linewidth}
\begin{verbatim}
  \begin{center}
\begin{tabular}{l|c|r}
 Uno & Dos & Tres \\
  \hline
 Cuatro & Cinco & Seis \\
Siete  & Ocho & Nueve\\
\hline
\end{tabular}
\end{center}
\end{verbatim}

\end{minipage}\hfill
\begin{minipage}[r]{0.48\linewidth}
  \begin{center}
\begin{tabular}{l|c|r}
 Uno & Dos & Tres \\
  \hline
 Cuatro & Cinco & Seis \\
Siete  & Ocho & Nueve\\
\hline
\end{tabular}
\end{center}

\end{minipage}

\end{exampleblock}


\end{frame}


\begin{frame}[fragile]
\frametitle{\textbf{Tablas}}
\justifying
\begin{itemize}
\item [] \verb+\begin{tabular}[pos]{formato}+
\item[] \verb+Tabla \end{tabular}+    
\end{itemize}
\begin{block}{Entorno Tabular}
\textbf{Formato}:
	\begin{description}
  \item [l] Alinear a la izquierda.
  \item [c] Centrar
  \item [r] Alinear a la derecha
  \item [p\{width\}] para párrafos
  \item [$|$] Genera una linea vertical
\end{description}

\textbf{Posición}:
\begin{description}
  \item [c] Centrada respecto al texto
  \item [t] Encima del texto
  \item [d] debajo del texto
\end{description}

\end{block}

\end{frame}

\begin{frame}[fragile]
\frametitle{\textbf{Tablas}}
\justifying
\begin{itemize}
\item [] \verb+\begin{tabular}[pos]{formato}+
\item[] \verb+Tabla \end{tabular}+    
\end{itemize}
\begin{block}{Entorno Tabular}
\textbf{Otros Comandos}:

\begin{description}
  \item [\&] pasa a la siguiente columna
  \item [$\textbackslash\textbackslash$] nueva fila
  \item [$\textbackslash$hline] genera una línea horizontal
  \item [$\textbackslash$cline{i-j}] genera una línea de la columna $i$ hasta la $j$
\end{description}

\end{block}

\end{frame}


\begin{frame}[fragile]
\frametitle{\textbf{Tablas}}
\justifying

\begin{exampleblock}{Ejemplo}
\begin{verbatim}
  \begin{center}
\begin{tabular}{|l|p{4cm}|r|} \hline \hline
Artículo & Descripción & Precio\\ \hline \hline
23543 & Laptop Sony Vaio VGN-NR21Z & 900 \\ \hline
64534 & Mouse inalámbrico Dell & 20 \\ \hline
\hline \end{tabular}
\end{center}
\end{verbatim}
\end{exampleblock}

\begin{block}{}
  \begin{center}
\begin{tabular}{|l|p{4cm}|r|} \hline \hline
Artículo & Descripción & Precio\\ \hline \hline
23543 & Laptop Sony Vaio VGN-NR21Z & 900 \\ \hline
64534 & Mouse inalámbrico Dell & 20 \\ \hline
\hline \end{tabular}
\end{center}	
\end{block}
\end{frame}


\begin{frame}[fragile]
\frametitle{\textbf{Tablas con Booktabs}}
\justifying
 \begin{itemize}\justifying
  \item Tablas de Aspecto Profesional.
  \item [] Requiere el paquete \texttt{booktabs}.
\end{itemize}

\begin{exampleblock}{Ejemplos}
\begin{minipage}[l]{0.48\linewidth}\small
\begin{table}[htb!]
\begin{tabular}[b]{llr}
\toprule
Animal & Description & Price (\$)\\
\midrule
Gnat & per gram & 13.65 \\
&each & 0.01 \\
Gnu & stuffed & 92.50 \\
Emu & stuffed & 33.33 \\ Armadillo & frozen & 8.99 \\
\bottomrule 
\end{tabular}
\caption{con booktabs}
\end{table}
\end{minipage}\hfill
\begin{minipage}[r]{0.48\linewidth}\small
\begin{table}[htb!]
\begin{tabular}[b]{llr}
\hline
Animal & Description & Price (\$)\\
\hline
Gnat & per gram & 13.65 \\
&each & 0.01 \\
Gnu & stuffed & 92.50 \\
Emu & stuffed & 33.33 \\ Armadillo & frozen & 8.99 \\
\hline 
\end{tabular}
\caption{sin booktabs}
\end{table}
\end{minipage}

\end{exampleblock}


\end{frame}

\begin{frame}[fragile]
\frametitle{\textbf{Tablas con Booktabs}}
\justifying
 \begin{block}{Comandos de booktabs}
\begin{description}\justifying\small
  \item [toprule] genera la línea superior de la tabla. Se pone justo al principio.
  \item [midrule] línea que delimita el comienzo de los datos de la tabla.
  \item [bottomrule] genera la línea inferior de la tabla.
  \item [cmidrule] es el comando análogo a \texttt{cline}, y dibuja una línea horizontal desde la columna $a$ a la columba $b$ que se le indique.
\end{description}

\end{block}

\begin{exampleblock}{}
\begin{center}
   \lstset{language=,basicstyle=\small}
   \vspace{-0.7cm}
\begin{lstlisting}
\begin{tabular}[b]{llr}
\toprule
Animal & Description & Price (\$)\\
\midrule
Gnat & per gram & 13.65 \\
&each & 0.01 \\
Gnu & stuffed & 92.50 \\
Emu & stuffed & 33.33 \\ Armadillo & frozen & 8.99 \\
\bottomrule 
\end{tabular}
\end{lstlisting}\vspace{-0.3cm}

\end{center}

\end{exampleblock}


\end{frame}

\begin{frame}[fragile]
\frametitle{\textbf{Figuras}}
\justifying
 \begin{itemize}\justifying
  \item Comando \texttt{includegraphics}.
  \item \verb+\usepackage{graphics}+
  \item \verb+\includegraphics[clave=valor]{archivo}]+
\end{itemize}

\begin{exampleblock}{}
\begin{minipage}[l]{0.6\linewidth}
   \lstset{language=}%SQL,basicstyle=\small}
   \vspace{-0.7cm}
\begin{lstlisting}
\includegraphics[width=6cm]{macros/ULAGOS}
\end{lstlisting}\vspace{-0.3cm}
\end{minipage}\hfill
\begin{minipage}[r]{0.28\linewidth}
\includegraphics[width=3cm]{macros/ULAGOS}
\end{minipage}
\end{exampleblock}
\end{frame}

\begin{frame}[fragile]
\frametitle{\textbf{Figuras}}
\justifying
 \begin{itemize}\justifying
  \item \verb+\includegraphics[clave=valor]{archivo}]+
  \item Opciones:
  \begin{description}\justifying
  \item [width] ancho de la imagen.
  \item [height] altura.
  \item [angle] permite rotar la imagen (en la dirección contraria a las agujas del reloj).
  \item [scale] para escalar la imagen.
\end{description}

\end{itemize}
\end{frame}


\begin{frame}[fragile]
\frametitle{\textbf{Entorno float}}
\justifying
 \begin{itemize}\justifying
  \item \verb+\begin{figure}[posicion]...\end{figure}+
  \item \verb+\begin{table}[posicion]...\end{table}+
  \item Posición:
  \begin{description}\justifying
  \item [h] intenta insertar el elemento ahí mismo.
  \item [t] en la parte superior de la página.
  \item [b] al final de la página.
  \item [p] en una página especial de tablas y figuras.
  \item [!] ignorar parámetros internos de \LaTeX{}, como el número máximo de figuras seguidas.
\end{description}
\item Se sigue un orden secuencial para insertar los elementos.
\item Si no se puede colocar uno, los demás se van acumulando.
\item User siempre más de un indicador de posición. Ejemplo \texttt{[htp!]}
\item El comando \texttt{clearpage} hace que se muestren todos los floats pendientes y se empiece una nueva página.
\end{itemize}

\end{frame}

\begin{frame}[fragile]
\frametitle{\textbf{Entorno float}}
\justifying\scriptsize
 \begin{itemize}\justifying
  \item Otras Funcionalidades:
  \begin{description}\justifying
  \item [$\textbackslash$caption\{\}] permite poner título a un float.\\ \texttt{$\textbackslash$caption[short]\{long\}}.
  \item [$\textbackslash$label\{\}] permite poner una etiqueta para referenciarlo.\\ \texttt{$\textbackslash$label\{etiqueta\}}.
  \item [$\textbackslash$listoffigures] genera un índice de figuras.
  \item [$\textbackslash$listoftables] genera un índice de tablas.
\end{description}

\end{itemize}

\begin{minipage}[l]{0.48\linewidth}\small
   \lstset{language=,basicstyle=\scriptsize}
   \vspace{-0.7cm}
\begin{lstlisting}
\begin{table}[htb!]
\caption{Tabla de ejemplo}
\label{tab:EjemploTablas1} \begin{tabular}{l|c|r}
\hline
Uno & Dos & Tres \\
Cuatro & Cinco & Seis \\ Siete & Ocho & Nueve \\
\hline \end{tabular}
\end{table}
\end{lstlisting}\vspace{-0.3cm}

\end{minipage}\hfill
\begin{minipage}[r]{0.48\linewidth}
\begin{table}[htb!]
\caption{Tabla de ejemplo}
\label{tab:EjemploTablas1} \begin{tabular}{l|c|r}
\hline
Uno & Dos & Tres \\
Cuatro & Cinco & Seis \\ Siete & Ocho & Nueve \\
\hline \end{tabular}
\end{table}
\end{minipage}

\begin{alertblock}{Importante}\scriptsize
Siempre pon el \texttt{caption} antes del \texttt{label}, primero le das el nombre y luego haces la referencia.
\end{alertblock}

\end{frame}

\section{Fórmulas Matemáticas}
\begin{frame}[fragile]
\frametitle{\textbf{Fórmulas Matemáticas}}
\justifying
 Incluir el paquete \texttt{amsmath}: \verb+\usepackage{amsmath}+
 \begin{block}{Tipos de fórmulas matemáticas}
\begin{description}\justifying
  \item [Entorno math] permite insertar fórmulas en una línea, así $x = \frac{y}{2}$. se puede abreviar con \$ formula \$.
  \item [Entorno displaymath] Para mostrar fórmulas en un párrafo a parte. Se puede abreviar con \texttt{$\textbackslash$[ $\textbackslash$]}. Ejemplo:
  \[x = \frac{y}{2}\]
  \item [Entorno equation] Permite introducir fórmulas numeradas a las que se pueden hacer referencias cruzadas.
\end{description}

\end{block}

\end{frame}

\begin{frame}[fragile]
\frametitle{\textbf{Fórmulas Matemáticas}}
\justifying
 \begin{exampleblock}{Ejemplo}
   \lstset{language=}%SQL,basicstyle=\small}
   \vspace{-0.7cm}
\begin{lstlisting}
\begin{displaymath}
\cos(2\theta) = \cos^2 \theta - \sin^2 \theta
\end{displaymath}
    
Como $\omega=2 \pi \cdot f$...
    
\begin{equation}
\frac{n!}{k!(n-k) !} = \binom{n}{k} \end{equation}
\end{lstlisting}\vspace{-0.3cm}

\end{exampleblock}
\begin{block}{}
\begin{displaymath}
\cos(2\theta) = \cos^2 \theta - \sin^2 \theta
\end{displaymath}
    
Como $\omega=2 \pi \cdot f$...

\begin{equation}
\frac{n!}{k!(n-k) !} = \binom{n}{k} \end{equation}
\end{block}

\end{frame}

\begin{frame}[fragile]
\frametitle{\textbf{Elementos habituales en fórmulas}}
\justifying
 \begin{description}\justifying
  \item [Subíndices] Se escriben con \_. Si el subíndice consta de más de un carácter, hay que encerrarlos entre llaves.\\ Ejemplo: \verb+X_i+ $\rightarrow X_i$, \verb+X_{21}+ $\rightarrow X_{21}$.
  \item [Superíndices] Se indican  con el carácter \^.\\ Ejemplo: \verb+X^2+ $\rightarrow X^2$, \verb+X^{21}+ $\rightarrow X^{21}$.
  \item [Fracciones] \verb+\frac{numerado}{denominador}+. Si numerador o denominador se componen solo de un número, no es necesario usar llaves.\\ Ejemplo: \verb+\frac 1 2+ $\rightarrow \frac{1}{2}$, \verb+\frac{X_{11}}{Y_{12}}+ $\rightarrow \frac{X_{11}}{Y_{12}}$
\end{description}

\end{frame}

\begin{frame}[fragile]
\frametitle{\textbf{Elementos habituales en fórmulas}}
\justifying
 \begin{description}\justifying
  \item [Signo multiplicación] el punto de multiplicación se escribe \verb+\cdot+\\Ejemplo: \verb+X\cdot Y+ $\rightarrow X \cdot Y$.\\Ojo: despues de cualquier comando (que empieza por $\textbackslash$) deja un espacio en blanco.
  \item [Más Menos] El signo más y el menos se escriben directamente $+$, $-$.\\ El signo más-menos se escribe de la siguiente forma: \verb+\pm+, y menos-más como \verb+\mp+\\Ejemplo: \verb+a \pm b+ $\rightarrow a \pm b$.
\end{description}

\end{frame}

\begin{frame}[fragile]
\frametitle{\textbf{Elementos habituales en fórmulas}}
\justifying
Signos Comparativos:
 \begin{description}\justifying
  \item [Menor]
\begin{itemize}\justifying
  \item \verb+<+ $\rightarrow <$
  \item $\textbackslash$ll $\rightarrow \ll$
  \item$\textbackslash$lll $\rightarrow \lll$
  \item$\textbackslash$leq $\rightarrow \leq$
  \item$\textbackslash$lneq $\rightarrow \lneq$
\end{itemize}
\item [Mayor]
\begin{itemize}\justifying
  \item \verb+>+ $\rightarrow >$
  \item$\textbackslash$gg $\rightarrow \ll$
  \item$\textbackslash$ggg $\rightarrow \lll$
  \item$\textbackslash$geq $\rightarrow \leq$
  \item$\textbackslash$gneq $\rightarrow \lneq$
\end{itemize}
\end{description}

\end{frame}
\begin{frame}[fragile]
\frametitle{\textbf{Elementos habituales en fórmulas}}
\justifying
 \begin{description}\justifying
  \item[Paréntesis, llaves, etc.] Se pueden escribir paréntesis, corchetes, llaves, etc. directamente, pero si queremos que se adapten al tamaño de la fórmula:\\
     \lstset{language=}%SQL,basicstyle=\small}
   \vspace{-0.7cm}
\begin{lstlisting}
\left(, \left[ y para cerrar \right), \right], etc.
\end{lstlisting}\vspace{-0.3cm}
Ejemplo: \verb+X_1 \cdot \left( \frac Y Z \right)+ $\rightarrow X_1 \cdot \left( \frac Y Z \right)$
\item [Letras Griegas]
  \texttt{$\textbackslash$alpha, $\textbackslash$beta, $\textbackslash$gamma, $\textbackslash$Gamma}
 $\rightarrow  \alpha  \beta \gamma \Gamma$
\item [Funciones] Algunas funciones están definidas como tales, y se pueden escribir en lugar de como texto:\\
Ejemplo: \texttt{$\textbackslash$sin x $\textbackslash$log 26} $\rightarrow \sin x log 26$
\end{description}

\end{frame}

\begin{frame}[fragile]
\frametitle{\textbf{Elementos habituales en fórmulas}}
\justifying
 \begin{description}\justifying
  \item[Texto] Inserción de espacio en blanco: \texttt{$\textbackslash$quad}.\\ Texto: \texttt{$\textbackslash$texto\{texto\}}.
\end{description}
\begin{exampleblock}{}
   \lstset{language=}%SQL,basicstyle=\small}
   \vspace{-0.7cm}
\begin{lstlisting}
\begin{displaymath}
Z_0 J_n=\frac{\pi \Delta}{2\sqrt{g_{n-1}\cdot g_n}} \quad \text{para n = 2, 3, \ldots, N} 
\end{displaymath}
\end{lstlisting}\vspace{-0.3cm}

\end{exampleblock}
\begin{block}{}
\begin{displaymath}
Z_0 J_n=\frac{\pi \Delta}{2\sqrt{g_{n-1}\cdot g_n}} \quad \text{para n = 2, 3, \ldots, N} 
\end{displaymath}
\end{block}

\end{frame}

\begin{frame}[fragile]
\frametitle{\textbf{Fórmulas Matemáticas}}
\justifying
 \begin{exampleblock}{}
   \lstset{language=}%SQL,basicstyle=\small}
   \vspace{-0.7cm}
\begin{lstlisting}
\begin{displaymath}
C_L=\frac{(S_{22}-\Delta S_{11}^*)^*}{|S_{22}|^2=-|\Delta|^2}
\end{displaymath}
    
\begin{displaymath}
R_S=\frac{\sqrt{1-g_s}\cdot (1-|S_{11}|^2)}{1-(1-g_s)\cdot|S_{11}|^2}
\end{displaymath}
\end{lstlisting}\vspace{-0.3cm}
\end{exampleblock}


\begin{block}{}
\begin{displaymath}
C_L=\frac{(S_{22}-\Delta S_{11}^*)^*}{|S_{22}|^2=-|\Delta|^2}
\end{displaymath}
\begin{displaymath}
R_S=\frac{\sqrt{1-g_s}\cdot (1-|S_{11}|^2)}{1-(1-g_s)\cdot|S_{11}|^2}
\end{displaymath}
\end{block}

\end{frame}

\begin{frame}[fragile]
\frametitle{\textbf{Referencias para Fórmulas en \LaTeX{}}}
\justifying
 \begin{description}\justifying
  \item [Wikibooks: \LaTeX{}] \url{https://es.wikibooks.org/wiki/Manual_de_LaTeX}
  \item [Manual extenso de símbolos] \url{http://www.ctan.org/tex-archive/info/symbols/comprehensive/symbols-a4.pdf}
  \item [Mathmode: Fórmulas matemáticas avanzadas] \url{http://www.tex.ac.uk/tex-archive/info/math/voss/mathmode/}
\end{description}

\end{frame}

\section{Personalización del Documento}

\begin{frame}[fragile]
\frametitle{\textbf{Personalización del Documento}}
\justifying
\begin{block}{Configurar el aspecto}
 \begin{itemize}\justifying
  \item Cuando trabajamos con \LaTeX{} \textbf{todo} es configurable.
  \item Podemos buscar cómo hacerlo en el manual de \LaTeX{} o en otros recursos online.
\end{itemize}
\end{block}

\begin{block}{Configurar los parámetros}
Se pueden cambiar parámetros de configuración del documento.
\begin{itemize}\justifying
  \item Para anular la identificación de párrafos:
  \begin{itemize}\justifying
  \item \texttt{$\textbackslash$setlength\{$\textbackslash$parindent\}\{0cm\}}
\end{itemize}
  \item Aumentar separación entre párrafos:
  \begin{itemize}\justifying
  \item \texttt{$\textbackslash$setlength\{$\textbackslash$parskip\}\{8pt\}}
\end{itemize}

\end{itemize}

\end{block}


\end{frame}

\begin{frame}[fragile]
\frametitle{\textbf{Cambiar nombre de capítulos, secciones, índices...}}
\justifying
 Se pueden cambiar los títulos de índices, imágenes, tablas, etc.
\begin{block}{}
    \lstset{language=}%SQL,basicstyle=\small}
   \vspace{-0.7cm}
\begin{lstlisting}
\renewcommand{\contentsname}{Contenido}
\renewcommand{\partname}{Parte}
\renewcommand{\indexname}{Lista Alfab\'etica}
\renewcommand{\appendixname}{Ap\'endice}
\renewcommand{\figurename}{Figura}
\renewcommand{\listfigurename}{Lista de Figuras}
\renewcommand{\tablename}{Tabla}
\renewcommand{\listtablename}{Lista de Tablas}
\renewcommand{\abstractname}{Resumen}
\renewcommand{\chaptername}{Cap\'itulo}
\renewcommand{\refname}{Bibliograf\'ia}
\end{lstlisting}\vspace{-0.3cm}
\end{block}


\end{frame}


\end{document}